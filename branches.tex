\section{Working with branches}

\begin{frame}

\frametitle{Branches in git}
	
\begin{enumerate}
	\item Branches are used to develop features in isolation from each other
	\item Git is specifically designed for efficient work with branches
	\item A branch is simply a pointer to a commit
	\item The default branch name is \alert{master}
\end{enumerate}	
		
\end{frame}

% -----------------------------------------------------------------------------

\begin{frame}[fragile]
	
\frametitle{Creating a branch}
	
\begin{block}{When do I need a branch?}
Generally, every time you start working on a new feature, or any significant change to your code.
\end{block}
	
Creating a branch:
\begin{minted}{console}
> git branch <branch name>
> git status
\end{minted}
	
Listing local branches:
\begin{minted}{console}
> git branch -v
\end{minted}	
	
Switching between branches:
\begin{minted}{console}
> git checkout <branch name>
\end{minted}
	
% TODO: add figures!
	
\end{frame}

% -----------------------------------------------------------------------------

\begin{frame}[fragile]

\frametitle{Working on a branch}
	
Switching back and forth between branches changes the files on your disk!

\begin{block}{Task: Bugfix on a local branch}
There is a bug in the \texttt{fact} function of the demo shell program. Create a new branch (give it a descriptive name) and check it out. Fix the bug. Commit.
\end{block}	

By default, branches are visible only locally. To make them visible at the origin, they must be pushed:
\begin{minted}{console}
> git push -u origin <branch name>
\end{minted}

\begin{block}{Task: New feature on a remote branch}
Implement a \texttt{square} command for the demo shell program, which computes the square of the provided argument. Create a new branch and check it out. Implement the feature. Commit. Push the branch to origin.
\end{block}	

\end{frame}

% -----------------------------------------------------------------------------

\begin{frame}[fragile]

\frametitle{Merging branches locally}

After you are finished working on the bugfix, it's time to merge it back into the master branch.
\begin{minted}{console}
> git checkout master
> git merge <branch name>
\end{minted}

After you are done with the bugfix branch, clean up after yourself by deleting it.
\begin{minted}{console}
> git branch -d <branch name>
\end{minted}
	
Merge conflicts are handled in the same way as discussed before. Remote branches at origin are also just branches, so we have been working with branches all along :) 

\end{frame}

% -----------------------------------------------------------------------------

\begin{frame}[fragile]
	
\frametitle{Pull requests}
	
\begin{block}{Task: Pull request}
Look for your branch in the web interface of the GitHub repo. Create a pull request for your branch. Have your partner review and merge the pull request. Delete the branch after it has been merged.
\end{block}	

\begin{block}{Code review}

Pull requests are an efficient and transparent code review mechanism. Code review is good. Pull requests are good. Use pull requests :)
\end{block}

\end{frame}

% -----------------------------------------------------------------------------

\begin{frame}[fragile]
	
\frametitle{Working with multiple remotes}
	
\begin{block}{Why would I need multiple remotes?}
We can get code changes from any repo, not just the one we originally cloned (which is called \texttt{origin}). A typical example is getting changes from the \texttt{upstream} repo, i.e., the repo that we forked.
\end{block}	

Listing and adding remotes:
\begin{minted}{console}
> git remote add <alias> git@<hostname>:<path to repo>	
> git remote -v	
\end{minted}
	
We can now work with the new remote in the same way as with origin (except for pushing!), e.g.:
\begin{minted}{console}
> git fetch <alias>
> git merge <alias>/<branch>
\end{minted}
	
\end{frame}

% -----------------------------------------------------------------------------

\begin{frame}

\frametitle{Merging upstream changes}

\begin{block}{Task: Merge upstream changes}
Create a remote called \texttt{upstream} pointing to \href{https://github.com/larics/git-tutorial-code.git}{the original repo you forked}. \texttt{fetch} the \texttt{upstream} repo and compare its \texttt{master} branch with your \texttt{master} branch. Merge the \texttt{upstream} \texttt{master} into your \texttt{master}.
\end{block}

\end{frame}

% -----------------------------------------------------------------------------
