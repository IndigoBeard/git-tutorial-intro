\section{Preparation}

\begin{frame}[fragile]

\frametitle{A note on notation}
	
Throughout the presentation, the following notation applies:

\begin{itemize}
	\item Commands that you are supposed to type are displayed in \texttt{monospace font} preceded by a \texttt{>} symbol, such as
	\begin{minted}{console}
> git --help
	\end{minted}
	\item The \texttt{>} symbol only indicates the command prompt.\\ \textit{Do not} type it in.
	\item When the command is too long to fit on one line, the line break will be marked with a backslash (\textbackslash), which you do not need to type.
	\item Text that you need to replace is given \texttt{<inside angle brackets>}, e.g.,
	\begin{minted}{console}
> cd /home/<your username>
	\end{minted}
	When typing the command with your replaced text,\\omit the brackets.
\end{itemize}

\end{frame}

% -----------------------------------------------------------------------------

\begin{frame}[fragile]

\frametitle{Preparing for the tutorial}
	
\begin{itemize}
	\item  Open a \href{https://github.com/join?source=header-home}{GitHub account} and set up an SSH key for passwordless login according to \href{https://help.github.com/articles/adding-a-new-ssh-key-to-your-github-account/}{these instructions}
	\item Install the \href{https://atom.io/}{Atom} text editor
	\item Install the git command-line client and GUI tools
	\begin{minted}{console}
> sudo apt install git git-gui gitg
	\end{minted}
	\item Tell git who you are, and which editor to use:
	\begin{minted}{console}
> git config --global user.name "<Your Name>"
> git config --global user.email <youremail@fer.hr>
> git config --global core.editor "atom --wait"
	\end{minted}
	\item The code example in the exercise will be available in C++ and Python. If you will be using C++, install the build tools:
	\begin{minted}{console}
> sudo apt install build-essential cmake
	\end{minted}
\end{itemize}

\end{frame}

% -----------------------------------------------------------------------------

\begin{frame}

\frametitle{Tutorial organization}
	
\begin{itemize}
	\item Work in pairs
	\item Pair up with someone using the same programming language (C++ or Python)
\end{itemize}
	
\end{frame}

% -----------------------------------------------------------------------------
