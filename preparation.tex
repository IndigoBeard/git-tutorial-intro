\section{Preparation}

\begin{frame}[fragile]
	\frametitle{Preparing for the tutorial}
	
	To follow the tutorial, you will need to:
	\begin{itemize}
		\item  Open a \href{https://github.com/join?source=header-home}{GitHub account}
		\item Set up an SSH key accordig to \href{https://help.github.com/articles/adding-a-new-ssh-key-to-your-github-account/}{these instructions}
		\item Install the git command-line client and GUI tools
		\begin{minted}{console}
		> sudo apt install git git-gui gitg
		\end{minted}
		\item Configure the command-line client
		\begin{minted}{console}
		> git config --global user.name "Your Name"
		> git config --global user.email youremail@fer.hr
		> git config --global core.editor \
		"gedit --wait --new-window"
		\end{minted}
	\end{itemize}
	
	\begin{block}{What's the SSH thing all about?}
	In short, it will allow you to communicate with the GitHub server without typing your GitHub password every time.
	\end{block}
\end{frame}

% -----------------------------------------------------------------------------

\begin{frame}
	\frametitle{Tutorial organization}
	
	\begin{itemize}
		\item Work in pairs
		\item Pair up with someone using the same programming language (C++ or Python)
	\end{itemize}
	
\end{frame}

% -----------------------------------------------------------------------------
